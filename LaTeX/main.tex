
\documentclass[10pt, a4paper, twoside]{report} % twoside=>para distinguir paginas pares de impares

\usepackage[utf8]{inputenc}
\usepackage[T1]{fontenc}
\usepackage[spanish, english]{babel}
\decimalpoint % cambia la coma por el punto como separador de decimales en math mode

%%%%%%%%%%%%%%%%%%%%%%%%%%%%%%%%%%%%%%%%%%%%%%%%%%%%%%%%%%%%%%%%%%%%
\usepackage{makeidx} % para el indice de palabras
\makeindex

%%%%%%%%%%%%%%%%%%%%%%%%%%%%%%%%%%%%%%%%%%%%%%%%%%%%%%%%%%%%%%%%%%%%
\usepackage{multirow}

%%%%%%%%%%%%%%%%%%%%%%%%%%%%%%%%%%%%%%%%%%%%%%%%%%%%%%%%%%%%%%%%%%%%
\usepackage[usenames, dvipsnames]{xcolor} % usenames y dvipsnames permite usar los colores predefinidos
\definecolor{gray97}{gray}{.97}
\definecolor{gray75}{gray}{.75}
\definecolor{gray45}{gray}{.45}

%%%%%%%%%%%%%%%%%%%%%%%%%%%%%%%%%%%%%%%%%%%%%%%%%%%%%%%%%%%%%%%%%%%%
\usepackage{amssymb} %para ciertos simbolos matemáticos
\usepackage{ dsfont }
\newtheorem{theorem}{Teorema}[section] %definicion de teoremas
\usepackage{amsmath} % para el \text (en modo matematico)

%%%%%%%%%%%%%%%%%%%%%%%%%%%%%%%%%%%%%%%%%%%%%%%%%%%%%%%%%%%%%%%%%%%%
\usepackage{listings} %para la inclusion de codigo fuente de lenguajes de programacion
\usepackage{etoolbox}
\newtoggle{InString}{}% Keep track of if we are within a string
\togglefalse{InString}% Assume not initally in string
\newcommand*{\ColorIfNotInString}[1]{\iftoggle{InString}{#1}{\color{red}#1}}%
\newcommand*{\ProcessQuote}[1]{#1\iftoggle{InString}{\global\togglefalse{InString}}{\global\toggletrue{InString}}}

\lstset{literate=%
    %{"}{{{\ProcessQuote{"}}}}1% Disable coloring within double quotes
    %{'}{{{\ProcessQuote{'}}}}1% Disable coloring within single quote
    {0}{{{\ColorIfNotInString{0}}}}1
    {1}{{{\ColorIfNotInString{1}}}}1
    {2}{{{\ColorIfNotInString{2}}}}1
    {3}{{{\ColorIfNotInString{3}}}}1
    {4}{{{\ColorIfNotInString{4}}}}1
    {5}{{{\ColorIfNotInString{5}}}}1
    {6}{{{\ColorIfNotInString{6}}}}1
    {7}{{{\ColorIfNotInString{7}}}}1
    {8}{{{\ColorIfNotInString{8}}}}1
    {9}{{{\ColorIfNotInString{9}}}}1
}

\lstset{ %
  backgroundcolor=\color{gray97},   % Indica el color de fondo
  basicstyle=\footnotesize,       % Fija el tamaño del tipo de letra utilizado para el código
  breakatwhitespace=false,        % Activarlo para que los saltos automáticos solo se apliquen en los espacios en blanco
  breaklines=true,                 % Activa el salto de línea automático
  captionpos=b,                    % Establece la posición de la leyenda del cuadro de código
  commentstyle=\color{gray45},    % Estilo de los comentarios
  deletekeywords={...},            % Si se quiere eliminar palabras clave del lenguaje
  escapeinside={\%*}{*)},          % Si quieres incorporar LaTeX dentro del propio código
  frame=shadowbox,	                   % Añade un marco al código
  keepspaces=true,                 % Mantiene los espacios en el texto. Es útil para mantener la indentación del código(puede necesitar columns=flexible).
  keywordstyle=\color{blue}\bfseries,       % estilo de las palabras clave
  language=Python,                 % El lenguaje del código
  otherkeywords={*,..., yield,True,False,self}, % Si se quieren añadir otras palabras clave al lenguaje
  numbers=left,                    % Posición de los números de línea (none, left, right).
  numbersep=5pt,                   % Distancia de los números de línea al código
  numberstyle=\small\color{gray45}, % Estilo para los números de línea
  rulecolor=\color{black},         % Si no se activa, el color del marco puede cambiar en los saltos de línea entre textos que sea de otro color, por ejemplo, los comentarios, que están en verde en este ejemplo
  showspaces=false,                % Si se activa, muestra los espacios con guiones bajos
  showstringspaces=false,          % subraya solamente los espacios que estén en una cadena de esto
  showtabs=false,                  % muestra las tabulaciones que existan en cadenas de texto con guión bajo
  stepnumber=2,                    % Muestra solamente los números de línea que corresponden a cada salto. 
  stringstyle=\ttfamily\color{green},       % Estilo de las cadenas de texto
  tabsize=2,	                   % Establece el salto de las tabulaciones a 2 espacios
  title=\lstname ,                   % muestra el nombre de los ficheros incluidos al utilizar
}
   
% minimizar fragmentado de listados
\lstnewenvironment{listing}[1][]
   {\lstset{#1}\pagebreak[0]}{\pagebreak[0]}

% estilo consola linux 
\lstdefinestyle{consola}
   {basicstyle=\scriptsize\bf\ttfamily,
    backgroundcolor=\color{gray75},}
    
%%%%%%%%%%%%%%%%%%%%%%%%%%%%%%%%%%%%%%%%%%%%%%%%%%%%%%%%%%%%%%%%%%%%%%%%%%
\usepackage{graphicx}
\graphicspath{{C:/Users/David/Desktop/TFG/TFGLatex/imagenes}}

%%%%%%%%%%%%%%%%%%%%%%%%%%%%%%%%%%%%%%%%%%%%%%%%%%%%%%%%%%%%%%%%%%%%%%%%%%
\usepackage{wrapfig} %usado en la licencia

%%%%%%%%%%%%%%%%%%%%%%%%%%%%%%%%%%%%%%%%%%%%%%%%%%%%%%%%%%%%%%%%%%%%%%%%%%
%\usepackage[top=2cm, bottom=2.25cm, outer=2.75cm, inner=2.75cm, 
%heightrounded, marginparwidth=2.5cm, marginparsep=0.3cm]{ geometry } %estilo de pagina
\usepackage[a4paper, top=2cm, bottom=2.25cm, outer=2.75cm, inner=2.75cm, 
heightrounded, marginparwidth=2.5cm, marginparsep=0.3cm]{ geometry }
%includehead,includefoot,

\usepackage{fancyhdr} % encabezados y pie de paginas

% clear default layout
\fancyhead{}
\fancyfoot{}

\fancyhead[LE,RO]{\textsc{\leftmark}}

\fancyfoot[C]{David Retana Ribeiro}
\fancyfoot[RE]{\thepage}
\fancyfoot[LO]{\thepage}

\renewcommand{\footrulewidth}{0.4pt}
\renewcommand{\headrulewidth}{0.4pt}
\pagestyle{fancy}

\fancypagestyle{plain}
  { % redefinimos plain porque se usa por defecto en las paginas de inicio de capitulo
  \fancyfoot[RE]{\thepage}
  \fancyfoot[LO]{\thepage}
  }

%%%%%%%%%%%%%%%%%%%%%%%%%%%%%%%%%%%%%%%%%%%%%%%%%%%%%%%%%%%%%%%%%%%%%%%%%
%\includeonly{G:/TFG/TFGLatex/1_capitulo/1_capitulo.tex} %lista de los fichero seccionados
%\include{}

%%%%%%%%%%%%%%%%%%%%%%%%%%%%%%%%%%%%%%%%%%%%%%%%%%%%%%%%%%%%%%%%%%%%%%%%%
\setcounter{secnumdepth}{3} %para que en el indice de contenidos enumere hasta las subsubsecciones

%%%%%%%%%%%%%%%%%%%%%%%%%%%%%%%%%%%%%%%%%%%%%%%%%%%%%%%%%%%%%%%%%%%%%%%%%
\usepackage[hidelinks]{hyperref} %para las referencias cruzadas
\hypersetup{
    colorlinks,
    %linkcolor={red!20!black},
    linkcolor={violet!50!black},
    citecolor={blue!50!black},
    urlcolor={blue!80!black}
}
%%%%%%%%%%%%%%%%%%%%%%%%%%%%%%%%%%%%%%%%%%%%%%%%%%%%%%%%%%%%%%%%%%%%%%%%%

\begin{document}
\selectlanguage{spanish}

%%%%%%%%%%%%%%%%%%%%%%%%%%%%%%%%%%%%%%%%%%%%%%%%%%%%%%%%%%%%%%%%%%
%%%%%%%%%%%%%%%%%%%%%%%%%%% PORTADA %%%%%%%%%%%%%%%%%%%%%%%%%%%%%%
%%%%%%%%%%%%%%%%%%%%%%%%%%%%%%%%%%%%%%%%%%%%%%%%%%%%%%%%%%%%%%%%%%

\input{C:/Users/David/Desktop/TFG/TFGLatex/portada/1_portada}

\newpage
$\ $
\thispagestyle{empty} % para que no se enumere esta pagina y no tenga header ni footer
\null\vfill
\noindent

%%%%%%%%%%%%%%%%%%%%%%%%%%%%%%%%%%%%%%%%%%%%%%%%%%%%%%%%%%%%%%%%%%
%%%%%%%%%%%%%%%%%%%% LICENCIA Y DEDICATORIAS %%%%%%%%%%%%%%%%%%%%%
%%%%%%%%%%%%%%%%%%%%%%%%%%%%%%%%%%%%%%%%%%%%%%%%%%%%%%%%%%%%%%%%%%
%\newpage
%$\ $
%\thispagestyle{empty} % para que no se enumere esta pagina
%
%\chapter*{}
%\pagenumbering{Roman} %comenzamos la enumeracion de paginas en numeros romanos
%\begin{flushright}
%  \textit{Dedicado a mi familia}
%\end{flushright}
%\newpage


\newpage
$\ $
\thispagestyle{empty} % para que no se enumere esta pagina
\null\vfill
\noindent

\pagenumbering{roman}

\textcopyright \hspace{0.3cm} Copyright $2017$ David Retana Ribeiro\\

\begin{wrapfigure}{r}{0.4\textwidth}
  \begin{center}
    %\href{https://creativecommons.org/licenses/by-sa/4.0/}
    \href{https://creativecommons.org/licenses/by-sa/3.0/es/}
         {\includegraphics[width=0.3\textwidth]{C:/Users/David/Desktop/TFG/TFGLatex/imagenes/cc-by-sa.png}}
  \end{center}
\end{wrapfigure}
\noindent Las imágenes contenidas en este documento \\
asi como el código fuente y la información \\
que recoge, se encuentran licenciadas \\
bajo una licencia \textit{Creative Commons}.

\clearpage

\begin{flushright}
  \textit{Dedicado a mi familia: \\
  Mis padres, Rosa y Pedro, por ese apoyo incondicional a lo largo de toda mi vida. \\
  Mis hermanos, Fátima y Carlos, por los buenos momentos.\\
  Y a todas aquellas personas que han pasado por mi vida.}
\end{flushright}
\newpage

%%%%%%%%%%%%%%%%%%%%%%%%%%%%%%%%%%%%%%%%%%%%%%%%%%%%%%%%%%%%%%%%%%
%%%%%%%%%%%%%%%% SOBRE EL AUTOR Y CONVENCIONES %%%%%%%%%%%%%%%%%%%
%%%%%%%%%%%%%%%%%%%%%%%%%%%%%%%%%%%%%%%%%%%%%%%%%%%%%%%%%%%%%%%%%%

\input{C:/Users/David/Desktop/TFG/TFGLatex/autor_convenciones/autor_convenciones}

%%%%%%%%%%%%%%%%%%%%%%%%%%%%%%%%%%%%%%%%%%%%%%%%%%%%%%%%%%%%%%%%%%
%%%%%%%%%%%%%%%%%%%%%% INDICES %%%%%%%%%%%%%%%%%%%%%%%%%%%%%%%%%%%
%%%%%%%%%%%%%%%%%%%%%%%%%%%%%%%%%%%%%%%%%%%%%%%%%%%%%%%%%%%%%%%%%%

\tableofcontents % imprime el indice de contenidos

%\cleardoublepage
\addcontentsline{lof}{chapter}{Indice de imagenes} % para que aparezca en el indice de contenidos
\listoffigures % indice de figuras

%\cleardoublepage
\addcontentsline{lot}{chapter}{Indice de tablas} % para que aparezca en el indice de contenidos
\listoftables % indice de tablas

\begingroup %para agruparlo con lo de arriba en la misma pagina
  \let\clearpage\relax
  %\cleardoublepage
  \addcontentsline{lol}{chapter}{Indice de códigos fuente} % para que aparezca en el indice de contenidos
  \lstlistoflistings % indice de codigos fuente
\endgroup

\clearpage

%%%%%%%%%%%%%%%%%%%%%%%%%%%%%%%%%%%%%%%%%%%%%%%%%%%%%%%%%%%%%%%%%%
%%%%%%%%%%%%%%%%%%%%% RESUMEN Y ABSTRACT %%%%%%%%%%%%%%%%%%%%%%%%%
%%%%%%%%%%%%%%%%%%%%%%%%%%%%%%%%%%%%%%%%%%%%%%%%%%%%%%%%%%%%%%%%%%

\input{C:/Users/David/Desktop/TFG/TFGLatex/resumen/resumen}


%%%%%%%%%%%%%%%%%%%%%%%%%%%%%%%%%%%%%%%%%%%%%%%%%%%%%%%%%%%%%%%%%%
%%%%%%%%%%%%%%%%%%%%%% INTRODUCCION %%%%%%%%%%%%%%%%%%%%%%%%%%%%%%
%%%%%%%%%%%%%%%%%%%%%%%%%%%%%%%%%%%%%%%%%%%%%%%%%%%%%%%%%%%%%%%%%%

\input{C:/Users/David/Desktop/TFG/TFGLatex/introduccion/introduccion}

%%%%%%%%%%%%%%%%%%%%%%%%%%%%%%%%%%%%%%%%%%%%%%%%%%%%%%%%%%%%%%%%%%
%%%%%%%%%%%%%%%%%%%%%%%%%%% PART 1 %%%%%%%%%%%%%%%%%%%%%%%%%%%%%%%
%%%%%%%%%%%%%%%%%%%%%% CAPITULO 1 %%%%%%%%%%%%%%%%%%%%%%%%%%%%%%%%
%%%%%%%%%%%%%%%%%%%%%%%%%%%%%%%%%%%%%%%%%%%%%%%%%%%%%%%%%%%%%%%%%%

\pagenumbering{arabic}
\part{Despliegue de un cluster Hadoop}
\input{C:/Users/David/Desktop/TFG/TFGLatex/apache_hadoop/apache_hadoop}

%%%%%%%%%%%%%%%%%%%%%%%%%%%%%%%%%%%%%%%%%%%%%%%%%%%%%%%%%%%%%%%%%%
%%%%%%%%%%%%%%%%%%%%%% CAPITULO 2 %%%%%%%%%%%%%%%%%%%%%%%%%%%%%%%%
%%%%%%%%%%%%%%%%%%%%%%%%%%%%%%%%%%%%%%%%%%%%%%%%%%%%%%%%%%%%%%%%%%

\input{C:/Users/David/Desktop/TFG/TFGLatex/instalacion_despliegue_cluster/instalacion_despliegue_cluster}

%%%%%%%%%%%%%%%%%%%%%%%%%%%%%%%%%%%%%%%%%%%%%%%%%%%%%%%%%%%%%%%%%%
%%%%%%%%%%%%%%%%%%%%%% CAPITULO 3 %%%%%%%%%%%%%%%%%%%%%%%%%%%%%%%%
%%%%%%%%%%%%%%%%%%%%%%%%%%%%%%%%%%%%%%%%%%%%%%%%%%%%%%%%%%%%%%%%%%

\input{C:/Users/David/Desktop/TFG/TFGLatex/computacion_paralela/computacion_paralela}


%%%%%%%%%%%%%%%%%%%%%%%%%%%%%%%%%%%%%%%%%%%%%%%%%%%%%%%%%%%%%%%%%%
%%%%%%%%%%%%%%%%%%%%%%%%% PART 2 %%%%%%%%%%%%%%%%%%%%%%%%%%%%%%%%%
%%%%%%%%%%%%%%%%%%%%% CAPITULOS 4 y 5 %%%%%%%%%%%%%%%%%%%%%%%%%%%%
%%%%%%%%%%%%%%%%%%%%%%%%%%%%%%%%%%%%%%%%%%%%%%%%%%%%%%%%%%%%%%%%%%

\part{Análisis de datos}
\input{C:/Users/David/Desktop/TFG/TFGLatex/analisis_datos/analisis_datos}

%%%%%%%%%%%%%%%%%%%%%%%%%%%%%%%%%%%%%%%%%%%%%%%%%%%%%%%%%%%%%%%%%%
%%%%%%%%%%%%%%%%%%%%%%% CONCLUSION %%%%%%%%%%%%%%%%%%%%%%%%%%%%%%%
%%%%%%%%%%%%%%%%%%%%%%%%%%%%%%%%%%%%%%%%%%%%%%%%%%%%%%%%%%%%%%%%%%

\input{C:/Users/David/Desktop/TFG/TFGLatex/conclusion/conclusion}

%%%%%%%%%%%%%%%%%%%%%%%%%%%%%%%%%%%%%%%%%%%%%%%%%%%%%%%%%%%%%%%%%%
%%%%%%%%%%%%%%%%%%%%%%% APENDICE %%%%%%%%%%%%%%%%%%%%%%%%%%%%%%%%%
%%%%%%%%%%%%%%%%%%%%%%%%%%%%%%%%%%%%%%%%%%%%%%%%%%%%%%%%%%%%%%%%%%

\part{Apéndice}
\appendix
\input{C:/Users/David/Desktop/TFG/TFGLatex/apendice/apendice}

%%%%%%%%%%%%%%%%%%%%%%%%%%%%%%%%%%%%%%%%%%%%%%%%%%%%%%%%%%%%%%%%%%
%%%%%%%%%%%%%%%%%%%%%% BIBLIOGRAFIA %%%%%%%%%%%%%%%%%%%%%%%%%%%%%%
%%%%%%%%%%%%%%%%%%%%%%%%%%%%%%%%%%%%%%%%%%%%%%%%%%%%%%%%%%%%%%%%%%

\addcontentsline{toc}{chapter}{Bibliografía} % para que lo añada al índice de contenidos
\nocite{*} % para que aparezcan todos los libros en la bibliografia sin ser citados explicitamente
\bibliography{bib_database}
\bibliographystyle{plain}

%%%%%%%%%%%%%%%%%%%%%%%%%%%%%%%%%%%%%%%%%%%%%%%%%%%%%%%%%%%%%%%%%%
%%%%%%%%%%%%%%%%%%%%% INDICE DE PALABRAS %%%%%%%%%%%%%%%%%%%%%%%%%
%%%%%%%%%%%%%%%%%%%%%%%%%%%%%%%%%%%%%%%%%%%%%%%%%%%%%%%%%%%%%%%%%%

\addcontentsline{toc}{chapter}{Índice alfabético} % para que lo añada al índice de contenidos
\printindex % para que ponga el índice aquí

\end{document}
